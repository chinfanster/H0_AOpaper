% mnras_template.tex 
%
% LaTeX template for creating an MNRAS paper
%
% v3.0 released 14 May 2015
% (version numbers match those of mnras.cls)
%
% Copyright (C) Royal Astronomical Society 2015
% Authors:
% Keith T. Smith (Royal Astronomical Society)
%
% v3.0 May 2015
%    Renamed to match the new package name
%    Version number matches mnras.cls
%    A few minor tweaks to wording
% v1.0 September 2013
%    Beta testing only - never publicly released
%    First version: a simple (ish) template for creating an MNRAS paper

%%%%%%%%%%%%%%%%%%%%%%%%%%%%%%%%%%%%%%%%%%%%%%%%%%
% Basic setup. Most papers should leave these options alone.
\documentclass[fleqn,usenatbib]{mnras}

% MNRAS is set in Times font. If you don't have this installed (most LaTeX
% installations will be fine) or prefer the old Computer Modern fonts, comment
% out the following line
\usepackage{newtxtext,newtxmath}
%\RequirePackage{amsfonts}
% Depending on your LaTeX fonts installation, you might get better results with one of these:
\usepackage{mathptmx}
%\usepackage{txfonts}
\usepackage{hyperref}
\usepackage{natbib}


% Use vector fonts, so it zooms properly in on-screen viewing software
% Don't change these lines unless you know what you are doing
\usepackage[T1]{fontenc}
\usepackage{ae,aecompl}


%%%%% AUTHORS - PLACE YOUR OWN PACKAGES HERE %%%%%

% Only include extra packages if you really need them. Common packages are:
\usepackage{graphicx}	% Including figure files
\usepackage{amsmath}	% Advanced maths commands
\usepackage{amssymb}	% Extra maths symbols

%%%%%%%%%%%%%%%%%%%%%%%%%%%%%%%%%%%%%%%%%%%%%%%%%%

%%%%% AUTHORS - PLACE YOUR OWN COMMANDS HERE %%%%%
\newcommand\rxj{RXJ\,1131$-$1231}
\newcommand\he{HE\,0435$-$1223}
\newcommand\pg{PG\,1115$+$080}
\newcommand{\sref}[1]{Section~\ref{#1}}
\newcommand{\fref}[1]{Figure~\ref{#1}}
\newcommand{\tref}[1]{Table~\ref{#1}}
\newcommand{\eref}[1]{Equation~(\ref{#1})}
\def\mathbi#1{\textbf{\em #1}}
\def\hst{\textit{HST}}
\def\zd{z_{\rm d}}
\def\zs{z_{\rm s}}
\def\dt{D_{\Delta t}}

% Please keep new commands to a minimum, and use \newcommand not \def to avoid
% overwriting existing commands. Example:
%\newcommand{\pcm}{\,cm$^{-2}$}	% per cm-squared

%%%%%%%%%%%%%%%%%%%%%%%%%%%%%%%%%%%%%%%%%%%%%%%%%%

%%%%%%%%%%%%%%%%%%% TITLE PAGE %%%%%%%%%%%%%%%%%%%

% Title of the paper, and the short title which is used in the headers.
% Keep the title short and informative.
\title[$H_{0} from 3 AO lensing imaging$]{SHARP - \uppercase\expandafter{\romannumeral 5}. $H_{0}$ from 3 strong-lensing-adaptive-optics imagings}


% The list of authors, and the short list which is used in the headers.
% If you need two or more lines of authors, add an extra line using \newauthor
\author[G.C.-F.~Chen et al.]{
Geoff~C.-F.~Chen,$^{1}$ \thanks{E-mail: chfchen@ucdavis.edu}
Cristian~E.~Rusu,$^{1}$
Christopher~D.~Fassnacht,$^{1}$
Sherry~H.~Suyu,$^{2,3}$
\newauthor{
Aleksi Halkola,
Matthew W.~Auger,$^{6}$
L$\acute{\text{e}}$on V.~E.~Koopmans,$^{7}$
David J.~Lagattuta,$^{8}$}
\newauthor{
John P.~McKean$^{7,9}$
and Simona Vegetti$^{2}$
}
\\
% List of institutions
$^{1}$Department of Physics, University of California, Davis, CA 95616, USA\\
$^{2}$Max Planck Institute for Astrophysics, Karl-Schwarzschild-Strasse 1, D-85740 Garching, Germany\\
$^{3}$Institute of Astronomy and Astrophysics, Academia Sinica, P.O.~Box 23-141, Taipei 10617, Taiwan\\
$^{4}$Institute of Astrophysics, National Taiwan University, Taipei 10617, Taiwan\\
$^{5}$Center for Theoretical Sciences, National Taiwan University, Taipei 10617, Taiwan\\
$^{6}$Institute of Astronomy, University of Cambridge, Madingley Rd, Cambridge, CB3 0HA, UK\\
$^{7}$Kapteyn Astronomical Institute, University of Groningen, P.O.Box 800, 9700 AV Groningen, The Netherlands\\
$^{8}$CRAL, Observatoire de Lyon, Universit Lyon 1, 9 Avenue Ch. Andr, F-69561 Saint Genis Laval Cedex, France\\
$^{9}$Netherlands Institute for Radio Astronomy (ASTRON), P.O. Box 2, 7990 AA Dwingeloo, The Netherlands \\
}

% These dates will be filled out by the publisher
\date{Accepted XXX. Received YYY; in original form ZZZ}

% Enter the current year, for the copyright statements etc.
\pubyear{2015}

% Don't change these lines
\begin{document}
\label{firstpage}
\pagerange{\pageref{firstpage}--\pageref{lastpage}}
\maketitle

% Abstract of the paper
\begin{abstract}
test
\end{abstract}

% Select between one and six entries from the list of approved keywords.
% Don't make up new ones.
\begin{keywords}
keyword1 -- keyword2 -- keyword3
\end{keywords}

%%%%%%%%%%%%%%%%%%%%%%%%%%%%%%%%%%%%%%%%%%%%%%%%%%

%%%%%%%%%%%%%%%%% BODY OF PAPER %%%%%%%%%%%%%%%%%%

\section{Introduction}
$\Lambda$
\citep{PerlmutterEtal99,RiessEtal98}




\section{Basic theory}
In this section, we briefly introduce the relation between the cosmology and the gravitational lensing and give the likelihood we use in \sref{sec:TDcosmo} and the joint inference of all information in \sref{sec:Jointinfer}.

\subsection{Time delay cosmography}
\label{sec:TDcosmo} % used for referring to this section from elsewhere
 When a background source is gravitationally lensed by  a foreground object, the light ray would be bended by the  gravitational field. The photon experience two components of time delay: (1) the geometry delay,  which is caused by different trajectory, and (2) the Shapiro delay, which is caused by the gravitational potential \citep{Shapiro64, Refsdal64}. We can express the total time delay as 
\begin{equation}
\label{eq:theory6}
t=\frac{\dt}{c}\left[\frac{1}{2}
\left(\boldsymbol{\theta}-\boldsymbol{\beta}\right)^{2}-\psi\left(\boldsymbol{\theta}\right)\right],
\end{equation}
where $\boldsymbol{\theta}$, $\boldsymbol{\beta}$, and
$\psi\left(\boldsymbol{\theta}\right)$ are the image coordinates, the source
coordinates, and the lens potential respectively.
The time delay distance is defined as
\begin{equation}
\label{eq:TDdistance}
\dt\equiv\left(1+
\zd\right)\frac{D_{\rm d}D_{\rm s}}{{D_{\rm ds}}}\propto H_{0}^{-1},
\end{equation}
$D_{\rm s}$ and $D_{\rm ds}$ are the angular diameter distances to the lens, to the source, and between the lens and the source, respectively. We use gravitational lensing with time delays to constrain the $D_{\rm ds}$, which is inversely proportional to the $H_{0}$.

In practice, we are only able to measure the time delay when the background source inside the strong lensing region, where the compact source can form mutiple images and the smooth source can form a extended images. We use compactly variable sources such as active galactic nuclei (AGN) to monitor the relative time delays, $\Delta t_{i,j}$, at $\theta_{i}$ and $\theta_{j}$ and use extended images (arc) to constrain the source position, $\beta$, as well as the potential, $\psi\left(\boldsymbol{\theta_{i}}\right), \psi\left(\boldsymbol{\theta_{j}}\right)$.




%\begin{equation}
%\Delta t_{i,j}=\frac{\dt}{c}\left[\frac{1}{2}
%\left(\boldsymbol{\theta_{i}}-\boldsymbol{\beta}\right)^{2}-\psi\left(\boldsymbol{\theta_{i}}\right)-\frac{1}{2}\left(\boldsymbol{\theta_{j}}-\boldsymbol{\beta}\right)^{2}+-\psi\left(\boldsymbol{\theta_{j}}\right)\right]
%\end{equation}
%-\psi\left(\boldsymbol{\theta_{j}}\right)\right],


If the background source is lensed by multiple deflectors at different redshift, we can model it through multi-plane lens equation. \citep[e.g.,][]{BlandfordNarayan86,SEF92,Collett&Auger14,McCullyEtal14} In this case, there is no single time-delay distance. However, if the lens system is dominated by a main lens such as \he, the time delay is primarily sensitive to the Equation \ref{eq:TDdistance}, with the deflector redshift as that of the primary strong lens plane. Therefore, we can define the effective time-delay distance, $\dt^{\textrm{eff}}\left(\zd,\zs\right)$. \citep{WongEtal16}

Since the light ray is affected by all the mass along the line of sight, we need to include the information from the environment (e.g., the mass contributed from cluster) in order to break the mass-sheet degeneracy, which can be generalized by source-position transformation. It can be seen as a degeneracies in the choice of the gravitational potential that leave all the observable quantities invariant except time delays. That is, if we don't count the mass along the line of sight, the inference of $\dt$ can be biased. If the effect from the environment are small, we can approximately the effect by the external convergence, $\kappa_{\textrm{ext}}$. The true time-delay distance can be expressed as
\begin{equation}
\dt^{\textrm{true}}=\frac{\dt^{\textrm{model}}}{1-\kappa_{\textrm{ext}}}
\end{equation}

Besides, although the mass enclosed in Einstein radius can be well-determined, the slope of the lens mass profile is only constrained near the extended images (arc). Since different mass profiles can exactly or approximately be mass sheet transformations of one form or another. $H_{0}$ can potentially be biased owing to the degeneracy between the slope and $D_{\Delta t}$. In order to break the degeneracy, we use Jeans equation \citep{Jeans1915J} and follow \citet{BinneyTremaine87} and \citet{SuyuEtal10,SuyuEtal13} to calculate the velocity dispersion, which can provide another information at different radius. 
%include the velocity dispersion information, 


By modeling the high resolution AO lens image and combining the time delays of the AGN, the external convergence from the environment, and the velocity dispersion of the lens, we can put a tight constraint on $H_{0}$. 


\subsection{Joint Inference}
\label{sec:Jointinfer}


\subsubsection{individual lens}
We follow \citet{WongEtal16} to give the inference of the $\dt$. We denote the $d_{i}$ for the imaging data, where $i=$ HE, RXJ, PG, represent \he, \rxj\, and \pg\ separately, $\Delta t_{i}$ for time delays, $d_{LOS_{i}}$ for LOS mass distribution, $\sigma_{i}$ for velocity dispersion. The $\xi_{i}$ is the parameters we want to infer from the data. A denote as the discrete assumption we made for the model.
The posterior of the $\xi_{i}$ can be expressed as
\begin{equation}
P(\xi_{i}|d_{i},\Delta t_{i},\sigma_{i}, d_{LOS_{i}}, A)\propto P(x)
\end{equation}


\subsubsection{combination of three lenses}


\section{Data}
\subsection{RXJ1131-1231}
The \rxj\ system was observed on the nights of UT 2012 May 16 and May 18 with the Near Infrared Camera 2 (NIRC2) on the Keck-2 Telescope \citep[e.g.,][]{wizinowich03}. This image was a part of SHARP data. The adaptive optics corrections were achieved through the use of a $R = 15.8$ tip-tilt star located 54.5 arcseconds from the lens system and a laser guide star.  The system was observed in the ``Wide Camera'' mode, which provides a roughly $40^{\prime\prime} \times 40^{\prime\prime}$ field of view and a pixel scale of 0.0397 arcseconds. This pixel scale slightly undersamples the point spread function (PSF), but the angular extent of the lens system and the distance from the tip-tilt star made the use of the Wide Camera the preferable approach.

The observations consisted of 61 exposures, each consisting of 6 coadded 10~s exposures, for a total on-source integration time of 3660~s.  The data were reduced by a python-based pipeline that has steps that do the flat-field correction, subtract the sky, correct for the optical distortions in the raw images, and combine the calibrated data frames \citep[for details, see][]{auger_eels}.  The final image has a pixel-scale of 0.04~arcseconds and is shown in Figure %\ref{fig:image}.
\subsection{HE0435-1223}
\subsection{PG1115+080}
\section{Lens modeling}
\subsection{Power-law mass model}
\subsection{Composite mass model}
\subsection{Kinematics}
We highlight the main steps here. The three dimensional can be expressed as 
\begin{equation}
\rho\left(r\right)=\left(\kappa_{\textrm{ext}}-1\right)\Sigma_{\textrm{crit}}\theta_{\textrm{E}}^{\gamma\prime-1}D_{\textrm{d}}^{\gamma\prime-1}\frac{\Gamma\left(\frac{\gamma\prime}{2}\right)}{\pi^{1/2}\Gamma\left(\right)}\frac{1}{r^{\gamma\prime}},
\end{equation}
\subsection{External convergence}

\subsection{RXJ1131-1231}
\subsubsection{systematics Tests}
\subsubsection{K}
\subsection{HE0435-1223}
\subsubsection{Blind systematics Tests}
\subsubsection{Unblind results}
\subsection{PG1115+080}
\subsubsection{Blind systematics Tests}
\subsubsection{Unblind results}
\section{Joint cosmological analysis}
\subsection{Cosmological inference from AO Strong Lensing}
\subsection{Cosmological inference comparison between AO and HST Strong Lensing image}
\subsection{Constraints in uniform cosmologies}
%\subsection{Constraining cosmological models beyond \texorpdfstring{$\Lambda$} CDM}
%\subsubsection{One-parameter extensions}
%\subsubsection{Two-parameter extensions}





\section{Conclusions}

\section*{Acknowledgements}

The Acknowledgements section is not numbered. Here you can thank helpful
colleagues, acknowledge funding agencies, telescopes and facilities used etc.
Try to keep it short.

%%%%%%%%%%%%%%%%%%%%%%%%%%%%%%%%%%%%%%%%%%%%%%%%%%

%%%%%%%%%%%%%%%%%%%% REFERENCES %%%%%%%%%%%%%%%%%%

% The best way to enter references is to use BibTeX:

\bibliographystyle{mnras}
\bibliography{AO_cosmography} % if your bibtex file is called example.bib


% Alternatively you could enter them by hand, like this:
% This method is tedious and prone to error if you have lots of references
%\begin{thebibliography}{99}
%\bibitem[\protect\citeauthoryear{Author}{2012}]{Author2012}
%Author A.~N., 2013, Journal of Improbable Astronomy, 1, 1
%\bibitem[\protect\citeauthoryear{Others}{2013}]{Others2013}
%Others S., 2012, Journal of Interesting Stuff, 17, 198
%\end{thebibliography}

%%%%%%%%%%%%%%%%%%%%%%%%%%%%%%%%%%%%%%%%%%%%%%%%%%

%%%%%%%%%%%%%%%%% APPENDICES %%%%%%%%%%%%%%%%%%%%%

\appendix

\section{Some extra material}

If you want to present additional material which would interrupt the flow of the main paper,
it can be placed in an Appendix which appears after the list of references.

%%%%%%%%%%%%%%%%%%%%%%%%%%%%%%%%%%%%%%%%%%%%%%%%%%


% Don't change these lines
\bsp	% typesetting comment
\label{lastpage}
\end{document}

% End of mnras_template.tex